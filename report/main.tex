\documentclass{report}

% Language setting
\usepackage[main=portuguese, english]{babel}
\usepackage{csquotes}

% Set page size and margins
\usepackage[a4paper,top=2cm,bottom=2cm,left=3cm,right=3cm,marginparwidth=1.5cm]{geometry}

% Useful packages
\usepackage{ulem}
\usepackage{parskip}
\usepackage{indentfirst}
\usepackage{setspace}
\usepackage{amsmath}
\usepackage{array}

\usepackage{graphicx}
\usepackage{xcolor}
\usepackage{colortbl}
\usepackage{subfigure}
\usepackage{titlesec}
\usepackage[colorlinks=false, allbordercolors={0 0 0}, pdfborderstyle={/S/U/W 0.25}]{hyperref}
\usepackage[hypcap=true]{caption}
\usepackage{enumitem}
\usepackage{soul}

\usepackage[
  backend=biber,
  style=apa,]{biblatex}
\addbibresource{reference.bib}

% Set section numbering from 1.1
\renewcommand{\thesection}{\arabic{section}.1}

\let\oldsection\section
\renewcommand\section{\clearpage\oldsection}

% Change section formatting
\titleformat{\section}
  {\fontsize{12}{15}\selectfont\bfseries}{\thesection}{1em}{}

% Configure indentations
\setlength{\parindent}{1.5cm}

\begin{document}

    \begin{titlepage}
        \centering
        
        \LARGE {Universidade Federal do Rio Grande do Sul \\ Instituto de Informática}
    
        \begin{figure}[h!]
        \centering
        \subfigure
        {\includegraphics[width=0.35\linewidth]{images/logos/UFRGS.png}}
        \hspace{1cm}
        \subfigure
        {\includegraphics[width=0.35\linewidth]{images/logos/INF.png}}
        \end{figure}
    
        \LARGE {INF01017 \\ Aprendizado de Máquina}
        
        \vfill
        {\noindent\hrulefill \\
        \bfseries \Huge{Trabalho Prático Final – Etapa 1} \\ \LARGE{Predição de Consumo de Combustível} \\
        \noindent\hrulefill}
        
        \vfill
        {\LARGE Luís Filipe Martini Gastmann (00276150) \\ Pedro Lubaszewski Lima (00341810) \\ Vinícius Boff Alves (00335551) \\~\\ Turma U}
    
        \vfill
        {\LARGE 9 de novembro de 2024}
        
    \end{titlepage}

        \renewcommand{\contentsname}{Sumário}
        \tableofcontents
        \clearpage
        \addtocontents{toc}{\protect\thispagestyle{empty}}

\section{Definição do Problema e Coleta de Dados}

O objetivo deste trabalho é prever o consumo médio de combustível de um carro através de algumas das suas características e origens de fabricação.
Alguns atributos das instâncias são a marca, a quantidade de cilíndros, o porte do veículo etc.

O conjunto de dados utilizado para desenvolver este trabalho foi obtido da seguinte página do Kaggle:
\href{https://www.kaggle.com/datasets/arslaan5/explore-car-performance-fuel-efficiency-data}{Explore Car Performance: Fuel Efficiency Data}.
Essa tarefa contará com diversas técnicas de preparação dos dados para posteriormente iniciar a seleção e
avaliação de modelos para essa tarefa.

\section{Análise Exploratória e Pré-processamento dos Dados}

\section{Abordagem, Algoritmos e Estratégias de Avaliação}

\section{\textit{Spot-checking} de Algoritmos}

\clearpage
%\printbibliography
\thispagestyle{empty}

\end{document}